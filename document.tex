%%This is a very basic article template.
%%There is just one section and two subsections.
\documentclass[12pt, a4paper]{article}

\usepackage[utf8]{inputenc} % Kodowanie UTF8
\usepackage{polski} % Wsparcie dla PL
\usepackage{listings}
\usepackage{color}
\usepackage[usenames,dvipsnames,svgnames,table]{xcolor}
\usepackage{hyperref} % dla linkowania do stron www

\begin{document}
\title{Esej na temat pakietu DBMS\_CRYPTO}
\author{Łukasz Szewczyk\\
Marek Lewandowski }
\date{\today}



\maketitle

\abstract{Esej wprowadza w najważniejsze zagadnienia pakietu, a
następnie
pokazuje widzę w praktyce na podstawie realnego przypadku zastosowania
pakietu}

\section{Wprowadzenie do pakietu}
%TODO: to co napisane jest w tym paragrafie jest to wyrzucenia/rozrzucenia
%po
%innych sekcjach
Pakiet DBMS\_CRYPTO zapewnia interfejs do szyfrowania i odszyfrowywania
danych w bazie danych oraz może być używany wraz z programami PL\/SQL
komunikującymi się przez sieć. Pakiet zapewnia wsparcie dla algorytmów
szyfrujących, które są uznane przez National Institute of Standards and
Technology, a jest to algorytm Rijndael spełniajacy standard AES. Gorzej
jest natomiast z algorytmami hashującymi, ponieważ pakiet nie zapewnia
wsparcia dla SHA-2, tylko dla SHA-1, które zostało już złamane, więc
nie powinno być uznawane za realizujące najwyższy możliwy poziom
bezpieczeństwa. Pakiet nie wspiera szyfrowania asymetrycznego oraz nie
generuje samemu kluczy dla algorytmów symetrycznych. Należy je
dostarczyć do funkcji pakietu samemu. Można jednak skorzystać z funkcji
pakietu do utworzenia klucza. Pakiet DBMS\_CRYPTO powstał jako zastępca
pakietu DBMS\_OBFUSCATION\_TOOLKIT, który jest już przestarzały ze
względu na bezpieczeństwo algorytmów i nie powinien być już używany.

% Pakiet DBMS\_CRYPTO to, jak można zgadywać po nazwie, pakiet
% realizujący pewne algorytmy z zakresu kryptografii. Można
%wykorzystywać pakiet do szyfrowania i
% deszyfrowania typów wbudowanych bazy Oracle włącznie z RAW, BLOB,
%CLOB. Pakiet
% posiada algorytm szyfru symetrycznego, który jest zgodny z obecnym
%standardem, a
% więc uważany jest za bezpieczny. Mowa tu o standardzie AES i
%algorytmie
% Rijndael, który kryje się pod standardem. Szyfr symetryczny jest szybki
%i nadaje
% się do szyfrowania bazy danych. Pakiet nie posiada natomiast żadnych
%szyfrów
% asymetrycznych.

\section{Obługiwane typy danych}
Funkcje pakietu operują na następujących typach danych: RAW, BLOB, CLOB.

\subsection{Jak szyfrować VARCHAR2?}
%TODO: Opis tego co nalezy zrobić żeby szyfrować VARCHAR2.
%you must convert it to the uniform database character set AL32UTF8, and
%then convert it to the RAW datatype. After 
%performing these conversions, you can then encrypt it with the DBMS_CRYPTO
%package.

Aby móc korzystać z funkcji pakietu dla typu VARCHAR2 należy wykonać
dwa kroki przygotowawcze.
\begin{enumerate}
	\item Należy zmienić kodowanie znaków na AL32UTF8
	\item Skonwertować otrzymany wynik na typ RAW
\end{enumerate}
Do wykonania tych czynności możemy posłużyć się pakietem UTL\_I18N.
 \lstset{language=SQL, basicstyle=\footnotesize, 
numberstyle=\footnotesize, stepnumber=1, numbersep=10pt, breaklines=true,
 caption = konwersja VARCHAR2 do RAW,
frame=shadowbox, rulesepcolor=\color{SkyBlue}}
\begin{lstlisting}
UTL_I18N.STRING_TO_RAW (string, 'AL32UTF8');
\end{lstlisting}

Proces odwrotny jest analogiczny, konwertujemy typ RAW na VARCHAR2, który
będzie kodowany w AL32UTF8, a następnie zmieniamy kodowanie na takie
jakie chcemy. Również tutaj możemy skorzystać z pakietu UTL\_I18N
podając za pierwszy argument typ RAW, a drugi nazwę typu kodowania.

 \lstset{language=SQL, basicstyle=\footnotesize, 
numberstyle=\footnotesize, stepnumber=1, numbersep=10pt, breaklines=true,
 caption = konwersja z RAW do VARCHAR2 o kodowaniu utf8,
frame=shadowbox, rulesepcolor=\color{SkyBlue}}
\begin{lstlisting}
UTL_I18N.RAW_TO_CHAR (data, 'utf8');
\end{lstlisting}

\section{Szyfrowanie symetryczne}
%TODO: co to jest, jakie wady i zalety
Szyfrowanie symetryczne to takie, w którym używamy jest ten sam klucz do
szyfrowania danych oraz do odszyfrowywania zaszyfrowanych danych. Klucz
powinien być wygenerowany z pseudolosowego generatora, który używa
źródła o dużej entropii do generowania liczb. Przykładem szyfrów
symetrycznych jest: DES, 3DES, Twofish, Serpent, IDEA. Dane dzielone są na
bloki, które podlegają szyfrowaniu. Różne algorytmy używają różnej
wielkości bloków. AES używa bloków o długości 128 bitów.
%TODO dodać linki do tych szyfrów
  
\paragraph{ Główną zaletą} szyfrowania symetrycznego jest to, że to
szyfrowanie jest o wiele szybsze w porównaniu do szyfrów asymetrycznych i
jest równie bezpieczne zakładając, że poprawnie realizowana jest
polityka bezpieczeństwa klucza.


\paragraph{ Główną wadą} jest to, że całe bezpieczeństwo opiera się
na tajności klucza, a więc jego transport jest bardzo wrażliwym
elementem całego systemu. Wymóg bezpiecznego wymieniania kluczy jest
zatem pewnym obciążeniem. Oprócz tego znana jest całkowita liczba
wszystkich kluczy.


%\subsection{Lista algorytmów szyfrujących}
%TODO: Tutaj lista algorytmów

\subsection{Tryby wiązania bloków}
Bloki w algorytmach szyfrujących mogą być szyfrowane w różny sposób.
Nazywa się to trybami łączenia bloków, np. tak aby blok był zależny
od poprzedniego bloku. Pakiet DBMS\_CRYPTO realizuje cztery główne tryby
łączenia bloków.
Są to:

\begin{description}
\item[ECB] Elektronic Codebook - każdy blok jest szyfrowany niezależnie
\item[CBC] Cipher Block Chaining - następny blok do zaszyfrowania jest
XOR'owany z poprzednio zaszyfrowanym blokiem
\item[CFB] Cipher-Feedback - pozwala na szyfrowanie danych mniejszych niż
rozmiar bloku
\item[OFB] Output Feedback - szyfr blokowy używany jest do wygenerowania
pseudolosowego ciągu danych, który następnie pełni role strumienia
szyfrującego, mieszanego z danymi za pomocą funkcji XOR
\end{description}

%TODO: linki do wiki

\subsection{Generowanie klucza}
Pakiet DBMS\_CRYPTO nie generuje samemu klucza. Można natomiast
skorzystać z funkcji \emph{RANDOMBYTES}, która wygeneruje losowy ciąg
bajtów o zadanej długości. Funkcja \emph{RANDOMBYTES} bazuje na RSA
X9.31 PRNG (Pseudo-RandomNumber Generator).
% NOTE: od wersji 11.2 to co poniżej zakomentowane jest nieprawdą.
% i używa źródła entropii z
% parametru SQLNET.CRYPTO\_SEED w pliku sqlnet.ora. Wartością domyślną
% parametru jest:
% 
%  \lstset{language=SQL, basicstyle=\footnotesize, 
% numberstyle=\footnotesize, stepnumber=1, numbersep=10pt, breaklines=true,
%  caption = domyślna wartość inicjalizacyjna generator,
% frame=shadowbox, rulesepcolor=\color{SkyBlue}}
% \begin{lstlisting}
% qwertyuiopasdfghjkl;zxcvbnm,.s1
% \end{lstlisting}

\lstset{language=SQL, basicstyle=\footnotesize, numbers=left,
numberstyle=\footnotesize, stepnumber=1, numbersep=10pt, breaklines=true,
 caption = Kod generujący 256 bitowy klucz,
frame=shadowbox, rulesepcolor=\color{SkyBlue}}
\begin{lstlisting}
DECLARE
dlugosc_klucza NUMBER := 256/8; -- 256 bitow (32 bajty)
klucz_raw RAW (32); -- 256 bitowy klucz
BEGIN
  klucz_raw := DBMS_CRYPTO.RANDOMBYTES(dlugosc_klucza);
END;
\end{lstlisting}

\subsection{Przechowywanie  i transport klucza}
Odpowiedzialność za bezpieczne przechowywanie i transport kluczy spada na developerów/użytkowników.

\section{Funkcje skrótu}
Służą do wyznaczenia krótkich sygnatur dla dowolnie dużych danych. 
Zmiana danych wejściowych zupełnie zmienia wartość skrótu, co umożliwia weryfikację integralności wiadomości.
\newline \newline
Własności funkcji skrótu:
\begin{enumerate}
  \item dla danej wiadomości łatwo policzyć skrót,
  \item mając skrót ciężko znaleźć wiadmość,
  \item ciężko znaleźć inną wiadomość o takim samym skrócie co dana wiadomość.
\end{enumerate}

\subsection{Dostępne funkcje}
\begin{description}
\item[SHA-1]Secure Hash Algorithm-- zaprojektowany przez NSA, prace chińskich naukowców doprowadziły wykrycia kolizji.
\item[MD4] Message-Digest algorithm 4-- aktualnie jest już bezużyteczny, zastępowany zazwyczaj przez MD5.
\item[MD5] Message-Digest algorithm 5-- popularna funkcja skrótu, która nie jest w pełni bezpieczna. W 2004 roku odkryto sposób na generowanie kolizji.
\end{description}

\section{Przypadek użycia - zabezpieczenie haseł użytkowników przy pomocy
\mbox{SHA-1}}
 Zaprezentujemy realny sposób przechowywania haseł użytkowników. Hasła nie będą
 przechowywanie w postaci jawnej lecz będą przechowywane skróty haseł. Podczas
 logowania, tworzony zostaje skrót hasła podanego przez użytkownika i sprawdzana
 jest poprawność tego hasła przez porównanie otrzymanego skrótu z tym zapisanym
 w bazie danych.
 
  \lstset{language=SQL, basicstyle=\footnotesize, numbers=left,
numberstyle=\footnotesize, stepnumber=1, numbersep=10pt, breaklines=true,
 caption = Procedura hashująca hasło,
frame=shadowbox, rulesepcolor=\color{SkyBlue}}
\begin{lstlisting}
create or replace
PROCEDURE hashowanie_hasla(haslo IN VARCHAR2, skrot OUT RAW)
IS
BEGIN
skrot:= DBMS_CRYPTO.hash(UTL_I18N.STRING_TO_RAW(haslo, 'AL32UTF8'), 3);
END;
\end{lstlisting}


  \lstset{language=SQL, basicstyle=\footnotesize, numbers=left,
numberstyle=\footnotesize, stepnumber=1, numbersep=10pt, breaklines=true,
 caption = Funkcja sprawdzająca poprawność hasła,
frame=shadowbox, rulesepcolor=\color{SkyBlue}}
\begin{lstlisting}
CREATE OR REPLACE FUNCTION LOGOWANIE(ID_UZYT NUMBER, HASLO VARCHAR2) RETURN CHAR
IS
SKROTTEMP RAW(20);
LICZBAWIERSZY NUMBER(2);
BEGIN
	HASHOWANIE_HASLA(HASLO, SKROTTEMP);

	SELECT COUNT (*) INTO LICZBAWIERSZY
	FROM SKROTY
	WHERE ID_UZYTKOWNIKA = ID_UZYT
      AND SKROT = SKROTTEMP;
      
	IF LICZBAWIERSZY = 0 THEN 
		RETURN 'N';
	ELSE
		RETURN 'T';
	END IF;
END;
\end{lstlisting}
 
  \lstset{language=SQL, basicstyle=\footnotesize, numbers=left,
numberstyle=\footnotesize, stepnumber=1, numbersep=10pt, breaklines=true,
 caption = test rozwiązania,
frame=shadowbox, rulesepcolor=\color{SkyBlue}}
\begin{lstlisting}
set serveroutput on format wraped
declare
ret char;
begin
ret := logowanie(4, 'pass4');
dbms_output.put_line(ret);
end;
\end{lstlisting} 


\section{Wady i zalety pakietu}
Pakiet udostępnia nam użyteczne algorytmy szyfrowania symetrycznego, pozwala generować losowe dane, na podstawie których można utworzyć klucz oraz udostępnia funkcje skrótów.
Stosunkowo niewielką wadą jest brak algorytmów asymetrycznych, gdyż duże ilości danych zazwyczaj szyfruje się symetrycznie. Poważną wadą jest jednak brak sposobu na bezpieczne przechowywanie oraz przesyłanie kluczy
wykorzystywanych do szyfrowania symetrycznego. Odpowiedzialność za to spada na developerów.

\section{Słowo końcowe}
Stworzenie przez ORACLE pakietu wspomagającego zabezpieczenie danych to dobry pomysł. Wydaje nam się jednak, że właściwym byłoby poszerzenie jego zakresu 
oraz częstsze uaktualnianie dostępnych funkcjonalności, np. wprowadzenie funkcji skrótu SHA-2.
\\\\\\
\emph{Źródła}:\\
\href{http://docs.oracle.com/cd/E11882_01/appdev.112/e25788/d_crypto.htm}{oracle
docs DBMS\_CRYPTO}\\
\href{http://en.wikipedia.org/wiki/Symmetric-key_algorithm}{Symmetric-key
algorithm}\\
\href{http://en.wikipedia.org/wiki/Block_cipher_modes_of_operation}{Block
chaining}
\href{http://en.wikipedia.org/wiki/Cryptographic_hash_function}{Cryptographic
hash function}

\end{document}
