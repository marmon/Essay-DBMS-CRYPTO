%%This is a very basic article template.
%%There is just one section and two subsections.
\documentclass[12pt, a4paper]{article}

\usepackage[utf8]{inputenc} % Kodowanie UTF8
\usepackage{polski} % Wsparcie dla PL
\usepackage{listings}
\usepackage{color}
\usepackage[usenames,dvipsnames,svgnames,table]{xcolor}
\usepackage{hyperref} % dla linkowania do stron www

\begin{document}
\title{Esej na temat pakietu DBMS\_CRYPTO}
\author{Łukasz Szewczyk\\
		Marek Lewandowski }
\date{\today}



\maketitle

\abstract{Esej wprowadza w najważniejsze zagadnienia pakietu, a następnie
pokazuje widzę w praktyce na podstawie realnege przypadku zastosowania pakietu}

\section{Wprowadzenie do pakietu}
Pakiet DBMS\_CRYPTO to, jak można zgadywać po nazwie, pakiet realizujący pewne
algorytmy z zakresu kryptografii. Można wykorzystywać pakiet do szyfrowania i
deszyfrowania typów wbudowanych bazy Oracle włącznie z RAW, BLOB, CLOB. Pakiet
posiada algorytm szyfru symetrycznego, który jest zgodny z obecnym standardem, a
więc uważany jest za bezpieczny. Mowa tu o standardzie AES i algorytmie
Rijndael, który kryje się pod standardem. Szyfr symetryczny jest szybki i nadaje
się do szyfrowania bazy danych. Pakiet nie posiada natomiast żadnych szyfrów
asymetrycznych.


\section{Mechanizm \ldots}


\section{Przypadek użycia - zabezpieczenie haseł użytkowników}
  
  \lstset{language=SQL, basicstyle=\footnotesize, numbers=left,
numberstyle=\footnotesize, stepnumber=1, numbersep=10pt, breaklines=true,
 caption = To jest opis kodu,
frame=shadowbox, rulesepcolor=\color{SkyBlue}} 
\begin{lstlisting}
CREATE OR REPLACE TRIGGER "KBD2A07"."PROJEKTY_AUTOINC" 
   before insert on "KBD2A07"."PROJEKTY" 
   for each row 
begin  
   if inserting then 
      if :NEW."ID_PROJEKTU" is null then 
         select PROJEKTY_ID_SEQ.nextval into :NEW."ID_PROJEKTU" from dual; 
      end if; 
   end if; 
end;
\end{lstlisting}
  

\end{document}